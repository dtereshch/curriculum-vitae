\documentclass[10pt]{article}

\usepackage[margin=3cm]{geometry}

\usepackage{array, xcolor}
\definecolor{lightgray}{gray}{0.8}
\newcolumntype{L}{>{\raggedleft}p{0.14\textwidth}}
\newcolumntype{R}{p{0.8\textwidth}}
\newcommand\VRule{\color{lightgray}\vrule width 0.5pt}

\usepackage{lipsum}

\usepackage{bibentry}

%%% Работа с русским языком
\usepackage{cmap}					% поиск в PDF
\usepackage{mathtext} 				% русские буквы в формулах
\usepackage[T2A]{fontenc}			% кодировка
\usepackage[utf8]{inputenc}			% кодировка исходного текста
\usepackage[english,russian]{babel}	% локализация и переносы

% \usepackage{fontspec}
% 	\setmainfont{Liberation Serif}
% 	\setmonofont{Liberation Mono}

%%% Работа с картинками
\usepackage{graphicx}  % Для вставки рисунков

% DOCUMENT LAYOUT
%\usepackage{geometry} 
\geometry{a4paper, textwidth=5.5in, textheight=8.5in, marginparsep=7pt, marginparwidth=.6in}
\setlength\parindent{0in}

\usepackage{hyperref}

% ---- MARGIN YEARS
\usepackage{marginnote}
\newcommand{\years}[1]{\marginnote{\scriptsize #1}}
\renewcommand*{\raggedleftmarginnote}{}
\setlength{\marginparsep}{7pt}
\reversemarginpar

\title{\bfseries\Huge Дмитрий Сергеевич Терещенко}
\author{Старший преподаватель Департамента экономики,\\Санкт-Петербургской школа экономики и менеджмента,\\Национальный исследовательский университет <<Высшая школа экономики>>}
\date{}


\begin{document}

\maketitle

\hrule
\section*{Контактная информация}
\noindent

	\textit{Электронная почта}: \\
	\href{mailto:dtereshchenko@hse.ru}{dtereshchenko@hse.ru}  • 
	\href{mailto:dtereshch@gmail.com}{dtereshch@gmail.com}\\
	
	\vspace{0.5em}

	\textit{Страницы в сети}: \\
	\href{https://www.hse.ru/staff/dtereshch}{Страница на сайте ВШЭ}  •
	\href{https://dtereshch.github.io/homepage/}{Личная веб-страница}  •
	\href{https://petrsu.ru/persons/2529/teretshenko}{Академия Google} •
	\href{https://elibrary.ru/author_items.asp?authorid=640419&pubrole=100&show_refs=1&show_option=0}{eLIBRARY}


\vspace{2em}
\hrule


%\section*{Текущая работа}
%\emph{Старший преподаватель}, кафедра экономики, управления производством и государственного и муниципального управления, Институт экономики и права, Петрозаводский государственный университет, г. Петрозаводск.

%\vspace{2em}
%\hrule

\section*{Образование}
\noindent
\years{2005-2010}\textit{экономика}, финансы и кредит (специалитет), Петрозаводский государственный университет, г. Петрозаводск. \\
%\years{2010-2013}\textit{экономическая теория} (аспирантура, без защиты диссертации), Петрозаводский государственный университет, г. Петрозаводск.

\vspace{2em}
\hrule

\section*{Опыт работы}
\noindent
\years{2018-\dots}\textit{Старший преподаватель}, Департамент экономики, Санкт-Петербургская школа экономики и менеджмента, Национальный исследовательский университет <<Высшая школа экономики>>.
\years{2022-\dots}\textit{Младший научный сотрудник}, Международный центр теории рынков и пространственной экономики, Национальный исследовательский университет <<Высшая школа экономики>>\\
\years{2019}\textit{Младший научный сотрудник}, Международный центр экономики, управления и политики в области здоровья, Национальный исследовательский университет <<Высшая школа экономики>>\\
\years{2017-2018}\textit{Старший преподаватель}, Институт экономики и права, Петрозаводский государственный университет.\\
\years{2012-2016}\textit{Преподаватель}, Экономический факультет, Петрозаводский государственный университет.\\
\years{2013-2015}\textit{Учитель экономики}, Державинский лицей, г. Петрозаводск.

\section*{Дисциплины}
\noindent

\textit{ВШЭ}

эконометрика • эмпирика отраслевых рынков • микроэкономика • основы экономики • цифровые навыки • модели дискретного выбора и потребительский выбор

\vspace{1em}

\textit{ПетрГУ}

экономика • деловые коммуникации • корпоративная социальная ответственность • экономика землепользования • экономика недвижимости • экономика землеустройства • экономика машиностроительного производства • экономика предприятия • экономика и организация технического сервиса • экономика и управление производством • основы и организация предпринимательской деятельности • маркетинг • экономико-правовые вопросы профессиональной деятельности • экономика и менеджмент высоких технологий • экологический аудит и~сертификация

%\vspace{2em}
%\hrule
%
%\section*{Научные интересы}
%институциональная экономика • экономический рост • региональные инновационные системы • экономическая роль университетов

\vspace{2em}
\hrule

\section*{Гранты, проекты, конкурсы}
\noindent
\years{2019-2020}\textit{Богатство российских субъектов Федерации и оценки эффективности государственного управления на региональном уровне} (грант РФФИ № 19-010-00324, исполнитель).\\
\years{2015-2017}\textit{Разработка интерактивной информационно-аналитической системы построения финансово-экономических моделей деятельности организации} (НИР, выполняемая при поддержке Фонда содействия инновациям в рамках конкурса <<УМНИК>>, соглашение № 6881ГУ/2015 от 30.07.2015).\\
\years{2013-2016}\textit{Приобретение профессиональных и предпринимательских навыков посредством воспитания предпринимательского духа и консультации начинающих предпринимателей} (проект ЕС Tempus).\\
\years{2015}\textit{Анализ развития инноваций и инновационного предпринимательства в Республике Карелия} (НИР, проводимая в рамках комплекса мероприятий Программы стратегического развития Петрозаводского государственного университета на 2012–2016 гг.).\\
\years{2013-2014}\textit{Влияние институциональных факторов на инвестиционный климат стран с транзитивной экономикой} (НИР, поддержанная Российским гуманитарным научным фондом, проект № 13-32-01269).\\
\years{2013}\textit{Оценка инновационного потенциала Карелии, связанного с производством и продвижением на рынок высокотехнологичной продукции} (НИОКР по заказу Министерства экономического развития Республики Карелия).\\
\years{2012}\textit{Экономический рост в странах Европейского Союза: значение институциональных факторов на современном этапе развития} (исследование, поддержанное грантом, предоставленным в рамках проекта <<Создание Центра ЕС в Баренц регионе России>>, осуществляемого при финансовой поддержке ЕС. Срок реализации: 21.12.2011-20.12.2014. Номер: 2011/279-218).\\




\vspace{2em}
\hrule

\section*{Повышения квалификации}
\noindent
\years{2022}\textit{Writing Academic Articles: Strengthening Your Chances for Publication}, НИУ ВШЭ (32 часа).\\
\years{2022}\textit{Эконометрика причинно-следственных связей}, Фонд Егора Гайдара (72 часа).\\
\years{2020}\textit{Представление информации и данных. Создание презентаций}, НИУ ВШЭ (24 часа).\\
\years{2019}\textit{Basics of Writing an Empirical Research Article}, НИУ ВШЭ (40 часов).\\
\years{2019}\textit{Empowering Your Academic Writing in English: Academic Writing vs General Writing}, НИУ ВШЭ (24 часа).\\
\years{2019}\textit{Что экономисту нужно знать о данных: избранные социально-экономические показатели}, Европейский университет и Фонд Егора Гайдара.\\
\years{2017-2019}\textit{Прикладные эконометрические методы для преподавателей-исследователей}, НИУ ВШЭ и Фонд Егора Гайдара (342 часа).\\
\years{2018}\textit{Программирование на языке Python для сбора и анализа данных}, Высшая школа экономики (32 часа).\\
\years{2018}\textit{Эконометрика от А до Я}, Высшая школа экономики (72 часа).\\
\years{2018}\textit{Обработка данных количественных исследований с помощью программных средств MS Excel и  IBM SPSS Statistic}, Петрозаводский государственный университет (36 часов).\\
\years{2016-2018}\textit{Современная микроэкономика}, Российская экономическая школа и Фонд Егора Гайдара (400 часов).\\
\years{2016}\textit{Культурное разнообразие и неравенство доходов}, Российская экономическая школа (52 часа).\\
\years{2016}\textit{Социокультурное и экономическое развитие регионов: восточный вектор}, Российская экономическая школа (72 часа).\\
\years{2016}\textit{Теория игр}, МГУ им. Ломоносова и Открытый университет Егора Гайдара (70 часов).\\
\years{2016}\textit{История экономических учений}, МГУ им. Ломоносова и Открытый университет Егора Гайдара (50 часов).\\
\years{2015}\textit{Культура, институты и разнообразие}, Российская экономическая школа (60 часов).\\
\years{2015}\textit{Эконометрика: введение в анализ временных рядов и панельных данных}, МГУ им. Ломоносова и Открытый университет Егора Гайдара (40 часов).\\
\years{2014-2015}\textit{Приобретение профессиональных и предпринимательских навыков посредством воспитания предпринимательского духа и консультации начинающих предпринимателей}, Венский экономический университет (72 часа).\\
\years{2014}\textit{Эконометрика. Вводный курс}, Фонд Егора Гайдара.\\
\years{2013-2015}\textit{Современная макроэкономика}, Российская экономическая школа и Фонд Егора Гайдара (350 часов).\\
\years{2013}\textit{Микроэкономика. Методика преподавания экономических дисциплин}, Фонд Егора Гайдара (72 часа).\\
\years{2013}\textit{Формирование профессиональных компетенций преподавателей вуза в области дистанционного обучения}, Петрозаводский государственный университет (72 часа).\\
\years{2013}\textit{Организация самостоятельной работы студентов в рамках компетентностной модели выпускника}, Петрозаводский государственный университет (24 часа).\\
\years{2012}\textit{Актуальные вопросы педагогики и психологии высшей школы}, Петрозаводский государственный университет (72 часа).\\
\years{2012}\textit{Современные педагогические технологии}, Петрозаводский государственный университет (72 часа).\\
\years{2011}\textit{Психология и педагогика высшей школы}, Петрозаводский государственный университет (72 часа).\\


\vspace{2em}
\hrule

\section*{Поощрения и награды}
\noindent
\years{2017}Благодарность Петрозаводского государственного университета \textit{за активное участие в процессах создания института экономики и права Петрозаводского государственного университета} (приказ № 938-лс).


\vspace{2em}
\hrule


\section*{Интеллектуальная собственность}
\noindent
\years{2017}Свидетельство о государственной регистрации программы для ЭВМ № 2017618624 <<Интерактивная информационно-аналитическая система построения финансово-экономических моделей>> (правообладатель --- Терещенко Д. С., автор --- Терещенко Д. С., заявка № 2017615744 от 19 июня 2017 г.).


\vspace{2em}
\hrule


\section*{Публикации}

\subsection*{Статьи в журналах (ВАК, РИНЦ)}
\noindent
\years{2023}Shcherbakov  V.  S., Tereshchenko  D.  S.  The  verification  of  regional  capital  mobility in the Russian Federation: A spatial econometric approach // St Petersburg University Journal of Economic Studies. 2023. Vol. 39. No. 1. P. 102–126. \url{https://doi.org/10.21638/spbu05.2023.105}.\\
\years{2021}Терещенко Д. С., Щербаков В. С. The Impact of Scientific Activity of Universities on Economic and Innovative Development // Economy of Region. 2021. Vol. 17. No. 1. P. 223-234.\\
\years{2021}Ореховский П. А., Терещенко Д. С., Щербаков В. С. Сменяемость руководства и экономическое развитие: существует ли взаимная связь? // Журнал институциональных исследований. 2021. Т. 13. № 1. С. 60-75.\\
\years{2020}Терещенко Д. С. Эмпирический анализ публикационной активности: конвергенция российских регионов? // Пространственная экономика. 2020. Т. 16. № 3. С. 109-138.\\
\years{2019}Терещенко Д. С., Щербаков В. С. Статистический анализ дифференциации российских регионов по уровню публикационной активности // ЭКО. 2019. № 9. С. 132-154.\\
\years{2019}Терещенко Д. С. Финансирование покупки недвижимости: выбор между ипотечным кредитованием и строительно-сберегательным кооперативом // Финансы и бизнес. 2019. Т. 15. № 4. С. 23-39.\\
\years{2019}Терещенко Д. С. Экономический анализ эффективности использования недвижимого имущества промышленных предприятий (на примере российских машиностроительных компаний) // Имущественные отношения в Российской Федерации. 2019. № 3 (210). С. 6-16.\\
\years{2019}Терещенко Д. С. Экономический анализ эффективности использования недвижимого имущества промышленных предприятий (на примере российских машиностроительных компаний). Окончание // Имущественные отношения в Российской Федерации. 2019. № 4 (211). С. 13-20.\\
\years{2018}Терещенко Д.С. Анализ динамики показателей регионального инновационного развития (на примере Республики Карелия) // Вестник Тюменского государственного университета. Социально-экономические и правовые исследования. 2018. Т. 4. № 2. С. 158-172. \\
\years{2018}Терещенко Д.С. Институциональные факторы инновационных процессов в российских регионах // Вестник Южно-Уральского государственного университета. Серия: Экономика и менеджмент. 2018. Т. 12. № 2. С. 55-62.\\
%\years{2018}Терещенко Д.С., Левкин Н.В. Теоретические вопросы построения системы институциональных факторов инновационного предпринимательства // Экономика и предпринимательство. 2018. № 8 (97). С. 709-712. 
%\years{2017}Терещенко Д.С., Коновалов А.П. Формирование финансово-экономической модели деятельности организации с учетом воздействия факторов внешней среды // Экономика и предпринимательство. 2017. №3-1. С. 1079-1084. \\
\years{2017}Терещенко Д.С., Щербаков В.С. Инновационные системы в экономическом развитии российских регионов: опыт эконометрического исследования роли вузов // Инновации. 2017. №5 (223). С. 75-83.\\
%\years{2017}Терещенко Д.С., Щербаков В.С. Экономический анализ публикационной активности в России // Экономика образования. 2017. №3 (100). С. 123-135. \\
\years{2016}Терещенко Д.С. Институциональные факторы экономического роста: идентификация, оценка и выявление нелинейности влияния // Вестник НГУЭУ. 2016. Т. 0. № 3. С. 299–314.\\
\years{2016}Терещенко Д.С., Щербаков В.С. Место и роль вузов в инновационном развитии регионов России // Региональная экономика: теория и практика. 2016. Т. 435. № 12. С. 165–177.\\
\years{2015}Терещенко Д.С., Щербаков В.С. Влияние экономических и политических институтов на инвестиционные процессы в регионе // Региональная экономика: теория и практика. 2015. Т. 408. № 33. С. 28–38.\\
\years{2014}Терещенко Д.С. Значение институциональных факторов для экономического роста стран транзитивного типа // Ученые записки Петрозаводского государственного университета. Серия: Общественные и гуманитарные науки. 2014. Т. 138. № 1. С. 104–107.\\
\years{2013}Терещенко Д.С. Институциональные факторы экономического роста: возможности оценки влияния // Предпринимательство. 2013. № 7. С. 33–45.\\
\years{2012}Терещенко Д.С. Экономический рост в странах БРИКС: роль и влияние институциональных факторов // Вестник Новосибирского государственного университета. Серия: Социально-экономические науки. 2012. Т. 12. № 3. С. 86–96.\\
\years{2012}Терещенко Д.С. Российская модель экономического роста: роль и влияние институциональных факторов в современных условиях // Экономика и экологический менеджмент. 2012. № 2. С. 471–497.\\
\years{2012}Терещенко Д.С. Институциональные факторы экономического роста России: проблема упущенных возможностей // Ученые записки Санкт-Петербургского университета управления и экономики. 2012. Т. 37. № 2. С. 62–70.\\
\years{2012}Терещенко Д.С. Международные индикаторы состояния системы институциональных факторов экономического роста // Корпоративное управление и инновационное развитие экономики Севера: Вестник Научно-исследовательского центра корпоративного права, управления и венчурного инвестирования Сыктывкарского государственного университета. 2012. № 4. С. 238–247.\\
\years{2012}Терещенко Д.С. К вопросу о количественной оценке институциональных факторов экономического роста // Экономика и экологический менеджмент. 2012. № 1. С. 413–421.\\
%\years{2012}Терещенко Д.С. Мировой опыт исследований взаимосвязи качества институтов и экономического роста: методический аспект // Экономика, предпринимательство и право. 2012. Т. 15. № 4. С. 3–12.\\
%\years{2012}Терещенко Д.С. Особенности институционального подхода к изучению экономического роста // Экономика, предпринимательство и право. 2012. Т. 13. № 2. С. 32–47. \\
%\years{2011}Терещенко Д.С. Модели устойчивого экономического роста, основанные на анализе институциональных факторов // Научные проблемы гуманитарных исследований. 2011. № 11. С. 281–292.\\

\subsection*{Препринты}
\noindent
\years{2022}Tereshchenko, Dmitrii, Competition among Russian Grocery Stores: Database on St. Petersburg, 2017-2021 (November 14, 2022). Higher School of Economics Research Paper No. WP BRP 258/EC/2022, Available at SSRN: \url{https://ssrn.com/abstract=4276261} or \url{http://dx.doi.org/10.2139/ssrn.4276261}.\\
\years{2021}Gaivoronskaia, Elizaveta and Iskhakov, Fedor and Karmeliuk, Maria and Kokovin, Sergey and Ozhegov, Evgeniy and Ozhegova, Alina and Tereshchenko, Dmitrii, Competition Among Russian Grocery Stores: Facts and Hypotheses to Explore (November 6, 2021). Available at SSRN: \url{http://dx.doi.org/10.2139/ssrn.3957743}. 

\subsection*{Монографии}
\noindent
\years{2019}Левкин Н. В., Терещенко Д. С. Идентификация и оценка институциональных факторов экономического роста стран транзитивного типа. Петрозаводск : Северный институт (филиал) ВГУЮ (РПА Минюста России).
\years{2014}Левкин Н.В., Терещенко Д.С. Институциональные аспекты формирования благоприятного инвестиционного климата в странах с экономикой транзитивного типа. Петрозаводск: Изд-во ПетрГУ, 2014. 124 с.

\subsection*{Учебные пособия}
\noindent
\years{2018}Гиенко Г. В., Годоева З. А., Конев И. П., Терещенко Д. С., Мурашкина Л. В., Резанова Л. В., Лаптев А. А. Экономика : учебное пособие для обучающихся по неэкономическим направлениям подготовки : [в 8 ч.] Ч. 5: Макроэкономика. Мировая экономика и международные экономические отношения.Петрозаводск: Издательство ПетрГУ, 220 с.\\
\years{2017}Коновалов А.П., Терещенко Д.С., Прокопьев Е.А. Основы инновационного менеджмента: учебное пособие для студентов вузов. Петрозаводск: Издательство ПетрГУ, 50 с.\\
\years{2015}Коновалов А.П., Терещенко Д.С. Экономика предпринимательства. Ч. 1. Ресурсы организации. Петрозаводск: Издательство ПетрГУ, 98 с.\\
\years{2015}Коновалов А.П. и др. Экономика предпринимательства. Ч. 2. Инновации, практика, документация. Петрозаводск: Издательство ПетрГУ, 81 с.\\


\subsection*{Сборники по итогам конференций}
\noindent
\years{2023}Терещенко, Д. Инновации и устойчивость институтов в российских регионах // Актуальные вопросы экономики и социологии : Сборник статей по материалам XIX Осенней конференции молодых ученых в Новосибирском Академгородке. Новосибирск: Институт экономики и организации промышленного производства СО РАН, 2023. С. 143–147.\\
\years{2018}Дусь Ю.П., Терещенко Д.С., Щербаков В.С. Сравнительный анализ международной студенческой мобильности в приграничных регионах Российской Федерации: case-study ОмГУ и ПетрГУ // Омские научные чтения [Электронный ресурс] : материалы Всероссийской научно-практической конференции (Омск, 10–15 декабря 2018 г.). Омск : Изд-во Ом. гос. ун-та, 2018. С. 783-785.\\
%\years{2018}Терещенко Д.С., Левкин Н.В. Экономические и институциональные предпосылки инновационного развития региона (на примере Республики Карелия) // Актуальные проблемы права и управления Сборник статей IV всероссийской научно-практической конференции с международным участием / Ответственный редактор А.А. Максимов. 2018. С. 394-403.\\
%\years{2018}Терещенко Д.С. Ресурсная зависимость российской экономики: значение институциональных факторов //  Актуальные проблемы права и управления: Сборник статей IV всероссийской научно-практической конференции с международным участием / Ответственный редактор А.А. Максимов. 2018. С. 385-393.\\
\years{2017}Дусь Ю.П., Щербаков В.С., Терещенко Д.С. Влияние государственных программ финансирования науки на публикационную активность университетов // Омские научные чтения [Электронный ресурс] : материалы Всероссийской научно-практической конференции (Омск, 11–16 декабря 2017 г.) / [редкол.: С. В. Белим и др.]. Электрон. текстовые дан. Омск : Изд-во Ом. гос. ун-та, 2017. 1 электрон. опт. диск (CD-ROM) ; 12 см. С. 575-578.\\
%\years{2017}Терещенко Д.С. Импортозамещение институтов в России: постановка исследовательского вопроса // Актуальные проблемы права и управления: сб. ст. III Всерос. науч.-практ. конф. (Петрозаводск, 20 мая 2017 г.) / под ред. А.А. Макисмова. Петрозаводск: Северный ин-т (филиал) ВГУЮ (РПА Минюста России), 2017. С. 380–386.\\
%\years{2017}Терещенко Д.С. Институциональные аспекты взаимодействия в БРИКС: противоречия и перспективы // Актуальные проблемы права и управления: сб. ст. II Всерос. науч.-практ. конф. (Петрозаводск, 3 декабря 2016 г.) / под ред. А.А. Макисмова. Петрозаводск: Северный ин-т (филиал) ВГУЮ (РПА Минюста России), 2017. С. 97–102. \\
%\years{2016}Терещенко Д.С. Система институциональных факторов экономического роста в теориях социального конфликта и общественного выбора // Наука и современность: сборник статей Международной научно-практической конференции (28 апреля 2016 г., г. Сызрань). Уфа: МЦИИ ОМЕГА САЙНС, 2016. С. 160–163.\\
%\years{2016}Терещенко Д.С. Влияние институциональных факторов на экономический рост стран транзитивного типа в рамках нелинейной модели зависимости (на примере группы БРИКС) // Прорывные научные исследования как двигатель науки: сборник статей Международной научно-практической конференции (3 мая 2016 г., г. Саранск). Уфа: МЦИИ ОМЕГА САЙНС, 2016. С. 150–154.\\
%\years{2015}Терещенко Д.С. Инновационная деятельность в Республике Карелия: оценка, динамика, факторы // Наука, образование и инновации: сборник статей Международной научно-практической конференции (28 декабря 2015 г., г. Челябинск). В 5 ч. Ч.2. Уфа: МЦИИ ОМЕГА САЙНС, 2015. С. 111–115.\\
\years{2015}Терещенко Д.С. Институциональные факторы экономического роста: теория социального конфликта и J-curve // Устойчивое развитие регионов: Материалы научно-практической конференции молодых учёных с международным участием в рамках Седьмого Молодежного экономического форума «Новая экономика – новые возможности», 12-14 ноября 2015 года, г. Петрозаводск. Петрозаводск: Карельский научный центр РАН, 2015. С. 238–248.\\
%\years{2015}Терещенко Д.С. Инструменты анализа инновационной деятельности в регионе (на примере Республики Карелия) // Современный взгляд на будущее науки: сборник статей Международной научно-практической конференции (28 октября 2015 г., г. Челябинск). В 2 ч. Ч.1. Уфа: РИО МЦИИ ОМЕГА САЙНС, 2015. С. 99–103.\\
%\years{2015}Терещенко Д.С., Коновалов А.П. Институты инновационного предпринимательства и их роль в региональном развитии // Взаимодействие науки и общества: проблемы и перспективы: сборник статей Международной научно-практической конференции (13 октября 2015 г., г. Уфа). Уфа: РИО МЦИИ ОМЕГА САЙНС, 2015. С. 164–168.\\
%\years{2015}Терещенко Д.С., Коновалов А.П. Инвестиционно-инновационный потенциал как основа развития региональной экономики // Проблемы и перспективы развития науки в России и мире: сборник статей Международной научно-практической конференции (8 октября 2015 г., г. Казань). Уфа: РИО МЦИИ ОМЕГА САЙНС, 2015. С. 102–106.\\
%\years{2015}Коновалов А.П., Терещенко Д.С. Оценка инновационного потенциала Республики Карелия // Влияние науки на инновационное развитие: сборник статей Международной научно-практической конференции (3 октября 2015 г., г. Самара). Уфа: РИО МЦИИ ОМЕГА САЙНС, 2015. С. 120–125.\\
\years{2014}Терещенко Д.С., Щербаков В.С. Региональные политические и экономические институты и их воздействие на инвестиционные процессы // Устойчивое развитие регионов: новая экономика – новые возможности: Материалы VI Молодежного экономического форума, 13–14 ноября 2014 года, г. Петрозаводск. Петрозаводск: Карельский научный центр РАН, 2014. С. 274–282.\\
%\years{2014}Левкин Н.В., Терещенко Д.С., Дружинин В.П. Динамика институтов формирования благоприятного инвестиционного климата в условиях трансформации рыночной модели хозяйства и квазистабильного состояния экономики // Научный взгляд на современное общество. Сборник статей Международной научно-практической конференции (28 сентября 2014 г., г. Уфа). Уфа: РИО МЦИИ ОМЕГА САЙНС, 2014. С. 59–62.\\
%\years{2014}Левкин Н.В., Терещенко Д.С. Система институциональных факторов экономического роста в современных российских условиях // Современные технологии управления–2014: сборник материалов международной научной конференции. Россия, Москва, 14-15 июля 2014 г. Киров: МЦНИП, 2014. С. 365–375.\\
%\years{2014}Левкин Н.В., Терещенко Д.С., Дружинин В.П. Экономический рост: теоретические предпосылки для базового анализа // Актуальные проблемы экономического развития: Сборник статей Международной научно-практической конференции (28 апреля 2014 г., г. Уфа). Уфа: Аэтерна, 2014. С. 138–141.\\
%\years{2014}Левкин Н.В., Терещенко Д.С. Страны БРИКС: возможное окончание глобального проекта // Закономерности и тенденции развития науки: сборник статей. Международная научно-практическая конференция, 27 марта 2014 г. Уфа: РИЦ БашГУ, 2014. С. 77–80.\\
\years{2013}Терещенко Д.С., Мордвинцев М.А., Реут О.Ч. Институты инновационного предпринимательства: исследовательский подход к изучению кластеризации // Международная экономическая интеграция: Материалы V Молодежного экономического форума, 14-15 ноября 2013 г. Петрозаводск: Карельский научный центр РАН, 2013. С. 184–190.\\
\years{2013}Терещенко Д.С. Экономический рост стран транзитивного типа: анализ с позиций институционального подхода // Международная экономическая интеграция: Материалы V Молодежного экономического форума, 14-15 ноября 2013 г. Петрозаводск: Карельский научный центр РАН, 2013. С. 68–75.\\
\years{2013}Терещенко Д.С. Институциональные факторы экономического роста в странах транзитивного типа // Институциональное развитие регионов в условиях модернизации российской экономики: материалы IV молодежной научно-практической конференции при поддержке Оксфордского российского фонда, 23-24 апреля 2013 года. Петрозаводск: Изд-во ПетрГУ, 2013. С. 72–75.\\
\years{2013}Терещенко Д.С. Экономический рост России: осуществление ретропрогноза на основе анализа институциональных факторов // Управление, наука, культура. Материалы 17-ой межрегиональной научно-практической конференции студентов и аспирантов (16-17 апреля 2013 г.). Петрозаводск: Карельский научный центр РАН, 2013. С. 236–238.\\
%\years{2013}Терещенко Д.С. Институциональные факторы экономического роста в различных группах стран в современных условиях // Составляющие научно-технического прогресса: Сборник материалов 9-й международной научно-практической конференции. Тамбов: ТМБпринт, 2013. С. 11–12.\\
%\years{2013}Терещенко Д.С. Институциональное отставание как фактор, сдерживающий экономический рост: анализ групп стран // Экономическое развитие страны: различные аспекты вопроса: Материалы IX Международной научно-практической конференции (26 марта 2013 г.): Сборник научных трудов. М.: Перо, 2013. С. 25–27.\\
\years{2012}Терещенко Д.С. Синергетический подход к анализу развития системы институциональных факторов экономического роста // Эволюционная и институциональная экономика: теория, методология, практика исследований: Материалы III Всероссийской летней школы молодых исследователей эволюционной и институциональной экономики при поддержке РГНФ, ОРФ (10-14 сентября 2012 года). Петрозаводск: Изд-во ПетрГУ, 2012. С. 219–227.\\
%\years{2012}Терещенко Д.С. Некоторые соображения о возможности применения синергетической методологии в исследованиях институциональных факторов экономического роста // Экономика XXI века: глобализация, кризисы, развитие: Материалы Международной научно-практической конференции (29-30 июня 2012 г., Харьков, Украина). Харьков: ИФИ, 2012. С. 107–113.\\
\years{2012}Терещенко Д.С. Институциональные факторы устойчивого экономического роста в странах переходного типа // Политико-правовое обеспечение устойчивого развития российского Севера и стран Северной Европы: история и современность: материалы межвуз. заоч. науч.-практ. конф. (Петрозаводск, 14 июня 2012 г.). Петрозаводск: Изд-во ПетрГУ, 2012. С. 183–196.\\
\years{2012}Терещенко Д.С. Изучение институциональных факторов экономического роста: применение синергетического подхода // Управление: история, наука, культура. Материалы XVI межрегиональной научно-практической конференции РАНХиГС. Петрозаводск: Карельский научный центр РАН, 2012. С. 155–158.\\
%\years{2012}Терещенко Д.С. Информационная открытость федеральных органов исполнительной власти в системе институциональных факторов экономического роста // Качество науки – качество жизни: Сборник материалов 8-ой международной научно-практической конференции, 28 февраля 2012. Тамбов: ТМБпринт, 2012. С. 3–4.\\
\years{2011}Терещенко Д.С. Устойчивое развитие региона как фактор роста национальной экономики // Институциональное развитие регионов в условиях модернизации российской экономики: материалы III молодеж. науч.-практ. конф. с междунар. участием при поддержке Оксфордского рос. фонда, Рос. гуманитар. науч. фонда (27-29 апреля 2011 года). : Изд-во ПетрГУ (Петрозаводск), 2011. С. 249–253.\\
\years{2011}Терещенко Д.С. Теоретические основы анализа системы институциональных факторов устойчивого экономического роста // Сборник материалов шестой и седьмой Школы молодых ученых «Социальная инноватика в региональном развитии». Петрозаводск: Карельский научный центр РАН, 2011. С. 295–301.\\
\years{2011}Терещенко Д.С. Система факторов устойчивого экономического роста // Конкурентоспособность российской экономики: Материалы III Молодежного экономического форума, 7-8 апреля 2011 года, г. Петрозаводск. Петрозаводск: Изд-во Карельского научного центра РАН, 2011. С. 41–45.\\
%\years{2009}Терещенко Д.С. Сделки слияния и поглощения в условиях мирового финансового кризиса // Гуманитарные ценности в экономике: Материалы научно-практической конференции. Петрозаводск: Изд-во ПетрГУ, 2009. С. 133–137.\\


\subsection*{Остальное}
\noindent
%Терещенко Д. С., Левкин Н. В. Институциональные факторы развития инновационных систем: теория и российская действительность // В кн.: Актуальные проблемы экономики и права: сборник трудов Вып. 2 (3). Межрегиональный центр инновационных технологий в образовании, 2019. С. 104-110.
%Левкин Н. В., Терещенко Д. С. Перспективы инновационного развития Республики Карелия: количественный и качественный анализ // В кн.: Актуальные проблемы экономики и права: сборник трудов Вып. 1 (2). Киров : Изд-во МЦИТО, 2019. С. 91-99.
%Левкин Н. В., Терещенко Д. С. Система факторов развития инноваций (региональный аспект) // В кн.: Актуальные проблемы экономики и права: сборник трудов. Киров : Изд-во МЦИТО, 2018. С. 61-67.
\years{2013}Терещенко Д.С. Экономический рост и институты: возможности оценки взаимосвязи // Модернизация региональной экономики. Труды Петрозаводского государственного университета. Серия: Экономика. Петрозаводск : Изд-во ПетрГУ, 2013. С. 161–169.\\
\years{2013}Терещенко Д.С. Экономический рост в странах Европейского Союза: значение институциональных факторов на современном этапе развития // Европейский Союз и Северная Европа: прошлое, настоящее и будущее: [сборник статей / проект <<Создание Центра ЕС в Баренц регионе России>>]. Петрозаводск: Петропресс, 2013. С. 24–45.\\
\years{2011}Мордвинцев М.А., Терещенко Д.С. Институциональные изменения в экономике региона в период глобального финансово-экономического кризиса (на примере Республики Карелия) // Модернизация региональной экономики. Труды Петрозаводского государственного университета. Серия: Экономика. Петрозаводск: Изд-во ПетрГУ, 2011. С. 78–86.\\



%\vspace{1cm}
\vfill{}
%\hrulefill

\begin{center}
	{\scriptsize  Дата составления: \today}
\end{center}

\end{document}
